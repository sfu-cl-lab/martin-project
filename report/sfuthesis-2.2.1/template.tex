%   DOCUMENT CLASS  %%%%%%%%%%%%%%%%%%%%%%%%%%%%%%%%%%%%%%%%%%%%%%%%%%%%%%%%%%%
%
%   Use the `sfuthesis` class to format your thesis. If your program does not
%   require a thesis defence, use the class option `undefended` like so:
%
%     \documentclass[undefended]{sfuthesis}
%
%   To generate a signature page for your defence, use the `sfuapproval` class
%   instead, by replacing the below line with
%
%     \documentclass{sfuapproval}
%
%   For more information about thesis formatting requirements, go to
%
%     http://www.lib.sfu.ca/help/publish/thesis
%
%   or ask a thesis advisor at the SFU Research Commons.
%

\documentclass{sfuthesis}



%   DOCUMENT METADATA  %%%%%%%%%%%%%%%%%%%%%%%%%%%%%%%%%%%%%%%%%%%%%%%%%%%%%%%%
%
%   Fill in the following information for the title page and approval page.
%

\title{An example of a thesis or dissertation on the subject of your degree}
\thesistype{Thesis}
\author{Stuart Arthur Dent}
\previousdegrees{%
	M.Sc., Wossamotta University, 1963\\
	B.Sc., Unseen University, 1836}
\degree{Doctor of Philosophy}
\discipline{Mathematics}
\department{Department of Inadvisably Applied Mathematics}
\faculty{Faculty of Mad Science}
\copyrightyear{2017}
\semester{Spring 2017}
\date{January 10, 2017}

\keywords{thesis template; Simon Fraser University; time travel paradoxes}

\committee{%
	\chair{Pamela Isely}{Professor}
	\member{Emmett Brown}{Senior Supervisor\\Professor}
	\member{Bonnibel Bubblegum}{Supervisor\\Associate Professor}
	\member{James Moriarty}{Supervisor\\Adjunct Professor}
	\member{Kaylee Frye}{Internal Examiner\\Assistant Professor\\School of Engineering Science}
	\member{Hubert J.\ Farnsworth}{External Examiner\\Professor\\Department of Quantum Fields\\Mars University}
}



%   PACKAGES %%%%%%%%%%%%%%%%%%%%%%%%%%%%%%%%%%%%%%%%%%%%%%%%%%%%%%%%%%%%%%%%%%
%
%   Add any packages you need for your thesis here.
%   You don't need to call the following packages, which are already called in
%   the sfuthesis class file:
%
%   - appendix
%   - etoolbox
%   - fontenc
%   - geometry
%   - lmodern
%   - nowidow
%   - setspace
%   - tocloft
%
%   If you call one of the above packages (or one of their dependencies) with
%   options, you may get a "Option clash" LaTeX error. If you get this error,
%   you can fix it by removing your copy of \usepackage and passing the options
%   you need by adding
%
%       \PassOptionsToPackage{<options>}{<package>}
%
%   before \documentclass{sfuthesis}.
%
%   The following packages are a few suggestions you might find useful.
%
%   (1) amsmath and amssymb are essential if you have math in your thesis;
%       they provide useful commands like ``blackboard bold'' symbols and
%       environments for aligning equations.
%   (2) amsthm includes allows you to easily change the style and numbering of
%       theorems. It also provides an environment for proofs.
%   (3) graphicx allows you to add images with \includegraphics{filename}.
%   (4) hyperref turns your citations and cross-references into clickable
%       links, and adds metadata to the compiled PDF.
%   (5) pdfpages lets you import pages of external PDFs using the command
%       \includepdf{filename}. You will need to do this if your research
%       requires an Ethics Statement.
%

\usepackage{amsmath}                            % (1)
\usepackage{amssymb}                            % (1)
\usepackage{amsthm}                             % (2)
\usepackage{graphicx}                           % (3)
\usepackage[pdfborder={0 0 0}]{hyperref}        % (4)
% \usepackage{pdfpages}                         % (5)
% ...
% ...
% ...
% ... add your own packages here!




%   OTHER CUSTOMIZATIONS %%%%%%%%%%%%%%%%%%%%%%%%%%%%%%%%%%%%%%%%%%%%%%%%%%%%%%
%
%   Add any packages you need for your thesis here. We've started you off with
%   a few suggestions.
%
%   (1) Use a single word space between sentences. If you disable this, you
%       will have to manually control spacing around abbreviations.
%   (2) Correct the capitalization of "Chapter" and "Section" if you use the
%       \autoref macro from the `hyperref` package.
%   (3) The LaTeX thesis template defaults to one-and-a-half line spacing. If
%       your supervisor prefers double-spacing, you can redefine the
%       \defaultspacing command.
%

\frenchspacing                                    % (1)
\renewcommand*{\chapterautorefname}{Chapter}      % (2)
\renewcommand*{\sectionautorefname}{Section}      % (2)
\renewcommand*{\subsectionautorefname}{Section}   % (2)
% \renewcommand{\defaultspacing}{\doublespacing}  % (3)
% ...
% ...
% ...
% ... add your own customizations here!




%   FRONTMATTER  %%%%%%%%%%%%%%%%%%%%%%%%%%%%%%%%%%%%%%%%%%%%%%%%%%%%%%%%%%%%%%
%
%   Title page, committee page, copyright declaration, abstract,
%   dedication, acknowledgements, table of contents, etc.
%
%   If your research requires an Ethics Statement, download one from the
%   SFU library website and uncomment the appropriate lines below.
%

\begin{document}

\frontmatter
\maketitle{}
\makecommittee{}

\begin{abstract}
	This is a blank document from which you can start writing your thesis.
\end{abstract}


\begin{dedication}
	This is an optional page.
\end{dedication}


\begin{acknowledgements}
	This is an optional page.
\end{acknowledgements}

\addtoToC{Table of Contents}%
\tableofcontents%
\clearpage

\addtoToC{List of Tables}%
\listoftables%
\clearpage

\addtoToC{List of Figures}%
\listoffigures%
\clearpage





%   MAIN MATTER  %%%%%%%%%%%%%%%%%%%%%%%%%%%%%%%%%%%%%%%%%%%%%%%%%%%%%%%%%%%%%%
%
%   Start writing your thesis --- or start \include ing chapters --- here.
%

\mainmatter%

\chapter{Introduction}

By default, only works cited in the text will be added to the bibliography~\cite{latexcompanion}.

\chapter{Results}

\section{Verifying Bellman Equation Agreement}

The action-value function $Q_\pi$ for a given policy $\pi$ over a Markov decision process satisfies a recursive relationship in terms of the current state-action pair $(s,a)$ and possible future state-action pairs $(S_{t+1},A_{t+1})$: 
\begin{align}
Q_\pi(s,a) &= \mathbb{E} \left[  R_{t+1} + \gamma Q_{\pi}(S_{t+1}, A_{t+1}) \, | \, S_t = s, A_t = a \right]\\
&=  R_{t+1} + \sum_{s' \in S, a' \in A} \gamma P(S_{t+1} = s', A_{t+1} = a' | S_t = s, A_t = a) Q_\pi(s',a').
\end{align}
This property is known as the Bellman equation \cite{sutton2018reinforcement} and it gives us the opportunity to verify the quality of our action-value estimates in an aspect that is different from what we did so far. If our approximation is indeed an action-value function, we expect it to (at least approximately) agree with the Bellman equation in addition to minimizing the loss function against the training data.\\\\
In our context we use a discount factor of $\gamma = 1$ and rewards only occur at episode-ending states. If $s$ is not a terminal state, $R_{t+1} = 0$ and the Bellman equation becomes
\begin{equation}
Q(s,a) = \sum_{s',a'} P(S_{t+1} = s', A_{t+1} = a' | S_t = s, A_t = a) Q(s',a').
\label{eq:markov_diff}
\end{equation}
Assuming we've computed an action-value function estimate for all the states in our dataset, we can proceed to verify this equality using the data, since the transition probability $P$ can be estimated by counting occurrences of states in the dataset (denoted by $c$) as
$$P(S_{t+1} = s', A_{t+1} = a' | S_t = s, A_t = a) = \frac{c(S_{t+1} = s', A_{t+1} = a', S_t = s, A_t = a)}{c(s,a)}.$$
Furthermore, since the state space is large, we wish to avoid examining a single state due to potential data sparsity effects. We will instead be interested in a set of state-action pairs $X$, summing all the corresponding state-action values as they appear in the dataset, namely the quantity:
$$\sum_{s,a\in X} c(s,a)Q(s,a).$$
Using \eqref{eq:markov_diff}, we obtain the right hand side:
\begin{equation}
\sum_{s,a\in X} c(s,a)Q(s,a) = \sum_{s,a\in X} c(s,a)\sum_{s',a'\in S\times A} \frac{c(s',a',s,a)}{c(s,a)} Q(s',a'),
\end{equation}
which simplifies to
\begin{equation}
\sum_{s,a\in X} c(s,a)Q(s,a) = \sum_{s,a\in X} \sum_{s',a'\in S\times A} c(s',a',s,a) Q(s',a').
\label{eq:markov_num}
\end{equation}
This final form is the one we use to compute the Bellman equation agreement, since both the left and right hand sides can be computed in a simple loop over the dataset sequences. We are interested in how much our action-value estimates violate \eqref{eq:markov_num}, i.e. the quantity
\begin{equation}
e_X = |\sum_{s,a\in X} c(s,a)Q(s,a) - \sum_{s,a\in X} \sum_{s',a'\in S\times A} c(s',a',s,a) Q(s',a') |.
\label{eq:markov_num2}
\end{equation}
We choose the state-action subset $X$ to include actions of a single type with only non-terminal outcomes (for example, we include all serve actions that do not result in an immediate error or point won). We compute the quantity $e_X$ for various action types and compare results across different action-value function approximations.

\section{Action Impact}

One of the motivations of computing the action-value function for sports data is the ability to numerically rank teams and players according to their action values. To achieve this, we need to consider that the context of an action performed by a player influences the quality of their actions. Namely, scoring a point from a favorable situation (eg. a successful attack following a perfect pass) should be treated differently than scoring in a situation that is deemed difficult. This leads us to the notion of action impact.\\\\
As discussed in \cite{routley2015markov}, there are several options of how we could choose to value actions. We will adopt the difference of consecutive action values as a measure of impact. Namely, for a transition from state $s$ to $s'$, with actions $a$ and $a'$ we define:
\begin{equation}
\text{impact}(s_t,a_t) = Q(s_t,a_t) - Q(s_{t-1},a_{t-1}).
\end{equation}
This allows us to capture to some degree how the flow of the game changed when action $a$ was performed. This version of valuing player actions was also used in the hockey context in \cite{liu2018deep}.

%   BACK MATTER  %%%%%%%%%%%%%%%%%%%%%%%%%%%%%%%%%%%%%%%%%%%%%%%%%%%%%%%%%%%%%%
%
%   References and appendices. Appendices come after the bibliography and
%   should be in the order that they are referred to in the text.
%
%   If you include figures, etc. in an appendix, be sure to use
%
%       \caption[]{...}
%
%   to make sure they are not listed in the List of Figures.
%

\backmatter%
	\addtoToC{Bibliography}
	\bibliographystyle{plain}
	\bibliography{msc_report}

\begin{appendices} % optional
	\chapter{Code}
\end{appendices}
\end{document}
